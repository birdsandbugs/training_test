% Options for packages loaded elsewhere
\PassOptionsToPackage{unicode}{hyperref}
\PassOptionsToPackage{hyphens}{url}
%
\documentclass[
]{article}
\usepackage{lmodern}
\usepackage{amssymb,amsmath}
\usepackage{ifxetex,ifluatex}
\ifnum 0\ifxetex 1\fi\ifluatex 1\fi=0 % if pdftex
  \usepackage[T1]{fontenc}
  \usepackage[utf8]{inputenc}
  \usepackage{textcomp} % provide euro and other symbols
\else % if luatex or xetex
  \usepackage{unicode-math}
  \defaultfontfeatures{Scale=MatchLowercase}
  \defaultfontfeatures[\rmfamily]{Ligatures=TeX,Scale=1}
\fi
% Use upquote if available, for straight quotes in verbatim environments
\IfFileExists{upquote.sty}{\usepackage{upquote}}{}
\IfFileExists{microtype.sty}{% use microtype if available
  \usepackage[]{microtype}
  \UseMicrotypeSet[protrusion]{basicmath} % disable protrusion for tt fonts
}{}
\makeatletter
\@ifundefined{KOMAClassName}{% if non-KOMA class
  \IfFileExists{parskip.sty}{%
    \usepackage{parskip}
  }{% else
    \setlength{\parindent}{0pt}
    \setlength{\parskip}{6pt plus 2pt minus 1pt}}
}{% if KOMA class
  \KOMAoptions{parskip=half}}
\makeatother
\usepackage{xcolor}
\IfFileExists{xurl.sty}{\usepackage{xurl}}{} % add URL line breaks if available
\IfFileExists{bookmark.sty}{\usepackage{bookmark}}{\usepackage{hyperref}}
\hypersetup{
  pdftitle={My practice paper template},
  pdfauthor={Graham Montgomery},
  hidelinks,
  pdfcreator={LaTeX via pandoc}}
\urlstyle{same} % disable monospaced font for URLs
\usepackage[margin=1in]{geometry}
\usepackage{color}
\usepackage{fancyvrb}
\newcommand{\VerbBar}{|}
\newcommand{\VERB}{\Verb[commandchars=\\\{\}]}
\DefineVerbatimEnvironment{Highlighting}{Verbatim}{commandchars=\\\{\}}
% Add ',fontsize=\small' for more characters per line
\usepackage{framed}
\definecolor{shadecolor}{RGB}{248,248,248}
\newenvironment{Shaded}{\begin{snugshade}}{\end{snugshade}}
\newcommand{\AlertTok}[1]{\textcolor[rgb]{0.94,0.16,0.16}{#1}}
\newcommand{\AnnotationTok}[1]{\textcolor[rgb]{0.56,0.35,0.01}{\textbf{\textit{#1}}}}
\newcommand{\AttributeTok}[1]{\textcolor[rgb]{0.77,0.63,0.00}{#1}}
\newcommand{\BaseNTok}[1]{\textcolor[rgb]{0.00,0.00,0.81}{#1}}
\newcommand{\BuiltInTok}[1]{#1}
\newcommand{\CharTok}[1]{\textcolor[rgb]{0.31,0.60,0.02}{#1}}
\newcommand{\CommentTok}[1]{\textcolor[rgb]{0.56,0.35,0.01}{\textit{#1}}}
\newcommand{\CommentVarTok}[1]{\textcolor[rgb]{0.56,0.35,0.01}{\textbf{\textit{#1}}}}
\newcommand{\ConstantTok}[1]{\textcolor[rgb]{0.00,0.00,0.00}{#1}}
\newcommand{\ControlFlowTok}[1]{\textcolor[rgb]{0.13,0.29,0.53}{\textbf{#1}}}
\newcommand{\DataTypeTok}[1]{\textcolor[rgb]{0.13,0.29,0.53}{#1}}
\newcommand{\DecValTok}[1]{\textcolor[rgb]{0.00,0.00,0.81}{#1}}
\newcommand{\DocumentationTok}[1]{\textcolor[rgb]{0.56,0.35,0.01}{\textbf{\textit{#1}}}}
\newcommand{\ErrorTok}[1]{\textcolor[rgb]{0.64,0.00,0.00}{\textbf{#1}}}
\newcommand{\ExtensionTok}[1]{#1}
\newcommand{\FloatTok}[1]{\textcolor[rgb]{0.00,0.00,0.81}{#1}}
\newcommand{\FunctionTok}[1]{\textcolor[rgb]{0.00,0.00,0.00}{#1}}
\newcommand{\ImportTok}[1]{#1}
\newcommand{\InformationTok}[1]{\textcolor[rgb]{0.56,0.35,0.01}{\textbf{\textit{#1}}}}
\newcommand{\KeywordTok}[1]{\textcolor[rgb]{0.13,0.29,0.53}{\textbf{#1}}}
\newcommand{\NormalTok}[1]{#1}
\newcommand{\OperatorTok}[1]{\textcolor[rgb]{0.81,0.36,0.00}{\textbf{#1}}}
\newcommand{\OtherTok}[1]{\textcolor[rgb]{0.56,0.35,0.01}{#1}}
\newcommand{\PreprocessorTok}[1]{\textcolor[rgb]{0.56,0.35,0.01}{\textit{#1}}}
\newcommand{\RegionMarkerTok}[1]{#1}
\newcommand{\SpecialCharTok}[1]{\textcolor[rgb]{0.00,0.00,0.00}{#1}}
\newcommand{\SpecialStringTok}[1]{\textcolor[rgb]{0.31,0.60,0.02}{#1}}
\newcommand{\StringTok}[1]{\textcolor[rgb]{0.31,0.60,0.02}{#1}}
\newcommand{\VariableTok}[1]{\textcolor[rgb]{0.00,0.00,0.00}{#1}}
\newcommand{\VerbatimStringTok}[1]{\textcolor[rgb]{0.31,0.60,0.02}{#1}}
\newcommand{\WarningTok}[1]{\textcolor[rgb]{0.56,0.35,0.01}{\textbf{\textit{#1}}}}
\usepackage{longtable,booktabs}
% Correct order of tables after \paragraph or \subparagraph
\usepackage{etoolbox}
\makeatletter
\patchcmd\longtable{\par}{\if@noskipsec\mbox{}\fi\par}{}{}
\makeatother
% Allow footnotes in longtable head/foot
\IfFileExists{footnotehyper.sty}{\usepackage{footnotehyper}}{\usepackage{footnote}}
\makesavenoteenv{longtable}
\usepackage{graphicx,grffile}
\makeatletter
\def\maxwidth{\ifdim\Gin@nat@width>\linewidth\linewidth\else\Gin@nat@width\fi}
\def\maxheight{\ifdim\Gin@nat@height>\textheight\textheight\else\Gin@nat@height\fi}
\makeatother
% Scale images if necessary, so that they will not overflow the page
% margins by default, and it is still possible to overwrite the defaults
% using explicit options in \includegraphics[width, height, ...]{}
\setkeys{Gin}{width=\maxwidth,height=\maxheight,keepaspectratio}
% Set default figure placement to htbp
\makeatletter
\def\fps@figure{htbp}
\makeatother
\setlength{\emergencystretch}{3em} % prevent overfull lines
\providecommand{\tightlist}{%
  \setlength{\itemsep}{0pt}\setlength{\parskip}{0pt}}
\setcounter{secnumdepth}{-\maxdimen} % remove section numbering

\title{My practice paper template}
\author{Graham Montgomery}
\date{2/7/2020}

\begin{document}
\maketitle

Dependencies

\begin{Shaded}
\begin{Highlighting}[]
  \KeywordTok{library}\NormalTok{(ggplot2)}
  \KeywordTok{library}\NormalTok{(knitr)}
  \KeywordTok{library}\NormalTok{(broom)}
  \KeywordTok{library}\NormalTok{(captioner)}
\end{Highlighting}
\end{Shaded}

Loading the data

\begin{Shaded}
\begin{Highlighting}[]
\KeywordTok{data}\NormalTok{(diamonds)}
\KeywordTok{head}\NormalTok{(diamonds)}
\end{Highlighting}
\end{Shaded}

\begin{verbatim}
## # A tibble: 6 x 10
##   carat cut       color clarity depth table price     x     y     z
##   <dbl> <ord>     <ord> <ord>   <dbl> <dbl> <int> <dbl> <dbl> <dbl>
## 1 0.23  Ideal     E     SI2      61.5    55   326  3.95  3.98  2.43
## 2 0.21  Premium   E     SI1      59.8    61   326  3.89  3.84  2.31
## 3 0.23  Good      E     VS1      56.9    65   327  4.05  4.07  2.31
## 4 0.290 Premium   I     VS2      62.4    58   334  4.2   4.23  2.63
## 5 0.31  Good      J     SI2      63.3    58   335  4.34  4.35  2.75
## 6 0.24  Very Good J     VVS2     62.8    57   336  3.94  3.96  2.48
\end{verbatim}

\hypertarget{abstract}{%
\section{Abstract}\label{abstract}}

I really like using R (R Core Team 2015) for science because of tools
like RStudio (RStudio Team 2015) and RMarkdown (RMarkdown Team 2015).
This document is a quick demonstration of writing an academic paper in
RMarkdown. There's a lot of other resources available on the web but
hopefully you'll find this document useful as an example.

\hypertarget{introduction}{%
\section{Introduction}\label{introduction}}

Writing reports and academic papers is a ton of work but a large amount
of that work can be spent doing monotonous tasks such as:

\begin{itemize}
\tightlist
\item
  Updating figures and tables as we refine our analysis
\item
  Editing our analysis and, in turn, editing our paper's text
\item
  Managing bibliography sections and in-text citations/references
\end{itemize}

These monotonous tasks are also highly error-prone. With RMarkdown, we
can close the loop, so to speak, between our analysis and our manuscript
because the manuscript can become the analysis.

As an alternative to Microsoft Word, RMarkdown provides some advantages:

\begin{itemize}
\tightlist
\item
  Free to use
\item
  Uses text so we can:

  \begin{itemize}
  \tightlist
  \item
    Use version control for

    \begin{itemize}
    \tightlist
    \item
      Tracking changes
    \item
      Collaborating
    \end{itemize}
  \item
    Edit it with our favorite and most powerful text editors
  \item
    Use the command line to for automation
  \end{itemize}
\end{itemize}

The rest of this document will show how we get some of the features we
need such as:

\begin{itemize}
\tightlist
\item
  Attractive typesetting for mathematics
\item
  Figures, tables, and captions
\item
  In-text citations
\item
  Bibliographies
\end{itemize}

\hypertarget{methods}{%
\section{Methods}\label{methods}}

Our analysis will be pretty simple. We'll use the \texttt{diamonds}
dataset from the \texttt{ggplot2} (Wickham 2009) package and run a
simple linear model. At the top of this document, we started with a code
chunk with \texttt{echo=FALSE} set as a chunk option so that we can load
the \texttt{ggplot2} package and \texttt{diamonds} dataset without
outputting anything to the screen.

For our analysis, we'll create a really great plot which really shows
the relationship between price and carat and shows how we include plots
in our document. Then we'll run a linear model of the form
\(y = mx + b\) on the relationship between price and carat and shows how
we include tables in our document. We can also put some more advanced
math in our paper and it will be beautifully typeset:

\[\sum_{i=1}^{N}{log(i) + \frac{\omega}{x}}\]

\[\int_{1}^{n}(x+3)\]

We can also use R itself to generate bibliographic entries for the
packages we use so we can give proper credit when we use other peoples'
packages in our analysis. Here we cite the \texttt{ggplot2} package:

\begin{Shaded}
\begin{Highlighting}[]
\OperatorTok{>}\StringTok{ }\KeywordTok{citation}\NormalTok{(}\StringTok{'ggplot2'}\NormalTok{)}

\NormalTok{To cite ggplot2 }\ControlFlowTok{in}\NormalTok{ publications, please use}\OperatorTok{:}

\StringTok{  }\NormalTok{H. Wickham. ggplot2}\OperatorTok{:}\StringTok{ }\NormalTok{Elegant Graphics }\ControlFlowTok{for}\NormalTok{ Data Analysis. Springer}\OperatorTok{-}\NormalTok{Verlag New York, }\FloatTok{2009.}

\NormalTok{A BibTeX entry }\ControlFlowTok{for}\NormalTok{ LaTeX users is}

  \OperatorTok{@}\NormalTok{Book\{,}
\NormalTok{    author =}\StringTok{ }\NormalTok{\{Hadley Wickham\},}
\NormalTok{    title =}\StringTok{ }\NormalTok{\{ggplot2}\OperatorTok{:}\StringTok{ }\NormalTok{Elegant Graphics }\ControlFlowTok{for}\NormalTok{ Data Analysis\},}
\NormalTok{    publisher =}\StringTok{ }\NormalTok{\{Springer}\OperatorTok{-}\NormalTok{Verlag New York\},}
\NormalTok{    year =}\StringTok{ }\NormalTok{\{}\DecValTok{2009}\NormalTok{\},}
\NormalTok{    isbn =}\StringTok{ }\NormalTok{\{}\DecValTok{978-0-387-98140-6}\NormalTok{\},}
\NormalTok{    url =}\StringTok{ }\NormalTok{\{http}\OperatorTok{:}\ErrorTok{//}\NormalTok{ggplot2.org\},}
\NormalTok{  \}}
\end{Highlighting}
\end{Shaded}

And then we just place that in our \texttt{.bibtex} file.

\hypertarget{results}{%
\section{Results}\label{results}}

The plot we made was really great (Figure 1).

\begin{Shaded}
\begin{Highlighting}[]
\KeywordTok{ggplot}\NormalTok{(diamonds, }\DataTypeTok{mapping =} \KeywordTok{aes}\NormalTok{(}\DataTypeTok{x =}\NormalTok{ carat,}
                               \DataTypeTok{y =}\NormalTok{ price,}
                               \DataTypeTok{color =}\NormalTok{ clarity)) }\OperatorTok{+}\StringTok{ }
\StringTok{           }\KeywordTok{geom_point}\NormalTok{() }\OperatorTok{+}\StringTok{ }
\StringTok{           }\KeywordTok{labs}\NormalTok{(}\DataTypeTok{x =} \StringTok{"Carat Weight"}\NormalTok{, }\DataTypeTok{y =} \StringTok{"Price ($)"}\NormalTok{)}
\end{Highlighting}
\end{Shaded}

\includegraphics{paper1_files/figure-latex/scatterplot-1.pdf}

\begin{Shaded}
\begin{Highlighting}[]
\NormalTok{mod <-}\StringTok{ }\KeywordTok{lm}\NormalTok{(price }\OperatorTok{~}\StringTok{ }\NormalTok{carat, diamonds)}
\KeywordTok{kable}\NormalTok{(}\KeywordTok{tidy}\NormalTok{(mod), }\DataTypeTok{digits =} \DecValTok{2}\NormalTok{)}
\end{Highlighting}
\end{Shaded}

\begin{longtable}[]{@{}lrrrr@{}}
\toprule
term & estimate & std.error & statistic & p.value\tabularnewline
\midrule
\endhead
(Intercept) & -2256.36 & 13.06 & -172.83 & 0\tabularnewline
carat & 7756.43 & 14.07 & 551.41 & 0\tabularnewline
\bottomrule
\end{longtable}

\hypertarget{discussion}{%
\section{Discussion}\label{discussion}}

\hypertarget{references}{%
\section*{References}\label{references}}
\addcontentsline{toc}{section}{References}

\hypertarget{refs}{}
\leavevmode\hypertarget{ref-RCoreTeam}{}%
R Core Team. 2015. ``R: A Language and Environment for Statistical
Computing.'' \url{http://www.r-project.org}.

\leavevmode\hypertarget{ref-RMarkdown}{}%
RMarkdown Team. 2015. \emph{Rmarkdown: R Markdown Document Conversion, R
Package}. Boston, MA: RStudio, Inc. \url{http://rmarkdown.rstudio.com/}.

\leavevmode\hypertarget{ref-RStudio}{}%
RStudio Team. 2015. \emph{RStudio: Integrated Development Environment
for R}. Boston, MA: RStudio, Inc. \url{http://www.rstudio.com/}.

\leavevmode\hypertarget{ref-ggplot}{}%
Wickham, Hadley. 2009. \emph{Ggplot2: Elegant Graphics for Data
Analysis}. Springer-Verlag New York. \url{http://ggplot2.org}.

\end{document}
